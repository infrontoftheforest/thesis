% !TEX root = ../main.tex
% Included by MAIN.TEX

%--------------------------------------------------
% Fonts and page setup
%--------------------------------------------------

% Default font
\usepackage{palatino}

\usepackage[utf8]{inputenc}

% Enable special PostScript fonts (optional)
% \usepackage{pifont}

% Manipulate the footer
\usepackage{scrlayer-scrpage}
\usepackage{scrhack}
\pagestyle{scrheadings}
\ifoot[\footertext]{\footertext} % \footertext set in INFO.TEX

% Set the font for the section headings
\renewcommand{\sectfont}{\normalfont \bfseries}

% Conditional commands in LaTeX documents, used for the \clearemptydoublepage.
\usepackage{ifthen}

% Typeset text in multiple columns (optional)
% \usepackage{multicol}

% Rotation tools, including rotated full-page floats (optional)
\usepackage{rotating}


%--------------------------------------------------
% Document structure
%--------------------------------------------------

% Pro­duce hy­per­text links in the doc­u­ment (recommended)
\usepackage{hyperref}

% Create glossaries and lists of acronyms
% depending on how many packages were shipped with your TeX distribution,
% you might need to install xindy. On Linux: sudo apt install xindy
\usepackage[toc, xindy]{glossaries}

% Standard LaTeX package for creating indexes
\usepackage{makeidx}


%--------------------------------------------------
% Bibliography
%--------------------------------------------------

% Set the bibliography style (default: plain)
\bibliographystyle{plain}

% Special biblography package (nice to have)
% \usepackage{natbib}


%--------------------------------------------------
% Graphics and floats
%--------------------------------------------------

% Enhanced support for graphics (recommended)
\usepackage{graphicx}
% Path to the figures directory (default: {figures/})
% Multiple entries are allowed, e.g. {{figures1/}{figures2/}}.
\graphicspath{{figures/}}

% Improved interface for floating objects (optional)
\usepackage{float}

% To use the subfigures (optional)
\usepackage{subcaption}


%--------------------------------------------------
% Mathematics
%--------------------------------------------------

% AMS mathematical facilities for LaTeX (recommended)
\usepackage{amsmath}

% TeX fonts from the American Mathematical Society (recommended)
\usepackage{amsfonts}

% Some extra math symbols (optional)
% \usepackage{amssymb}

% Extended maths fonts for LaTeX (optional)
% \usepackage{yhmath}

% Provide math delimiters whose size can be computed automatically (optional)
% \usepackage{commath}


%--------------------------------------------------
% Source code and algorithms
%--------------------------------------------------

% Source code typesetting
% \usepackage{listings} % (optional - alternative)
\usepackage[newfloat]{minted} % (recommended)
% Set global Minted options
\setminted{linenos, autogobble, frame=lines, framesep=2mm}
% Inline C++ (optional)
\newcommand{\incpp}[1]{\mintinline{c++}{#1}}
\newenvironment{code}{\captionsetup{type=listing}}{}
\SetupFloatingEnvironment{listing}{name=Source Code}

% Typeset algorithms - pseudocode (optional)
% \usepackage{algorithmicx}
% \usepackage{algpseudocode}
% Normal arrow comments
% \algrenewcommand{\algorithmiccomment}[1]{\hfill$\rightarrow$ #1}


%--------------------------------------------------
% Tables
%--------------------------------------------------

% Tables (optional)
\usepackage{tabu}

% Add color to LaTeX tables (optional)
% \usepackage{colortbl}

% Create tabular cells spanning multiple rows (optional)
% \usepackage{multirow}


%--------------------------------------------------
% Color
%--------------------------------------------------

% Use colors
\usepackage[dvipsnames]{xcolor}

% You may find all the pre-defined colors in
% https://en.wikibooks.org/wiki/LaTeX/Colors#Predefined_colors

% Custom colors
\definecolor{Pantone300C}{HTML}{0065BD} % TUM primary blue
\definecolor{Pantone301}{HTML}{005293}  % TUM secondary light blue
\definecolor{Pantone540}{HTML}{003359}  % TUM secondary dark blue
\definecolor{DarkGray}{HTML}{333333}    % TUM secondary dark gray
\definecolor{MediumGray}{HTML}{808080}  % TUM secondary medium gray
\definecolor{LightGray}{HTML}{CCCCC6}   % TUM secondary light gray
\definecolor{Pantone7527}{HTML}{DAD7CB} % TUM accent gray
\definecolor{Pantone158}{HTML}{E37222}  % TUM accent orange
\definecolor{Pantone383}{HTML}{A2AD00}  % TUM accent green
\definecolor{Pantone283}{HTML}{98C6EA}  % TUM accent very light blue
\definecolor{Pantone542}{HTML}{64A0C8}  % TUM accent light blue

% Color for the hyperlinks (e.g. table of contents)
\def\colorLinks{Pantone300C}
% Color for the web links
\def\colorUrl{Pantone542}
% Color for the citations
\def\colorCitations{Pantone158}

%--------------------------------------------------
% PDF output
%--------------------------------------------------

% Adjust the color of the links
\hypersetup{
  linkcolor=\colorLinks,%
  urlcolor=\colorUrl,%
  citecolor=\colorCitations
}

% Disable the coloring of the links when printing.
% Requires a compatible PDF reader.
\usepackage[ocgcolorlinks]{ocgx2}[2017/03/30]

% PDF Metadata
\hypersetup{
  pdftitle={\metaTitle},%
  pdfauthor={\metaAuthor},%
  pdfkeywords={\metaKeywords},%
  pdfsubject={\metaSubject}
}

% Create XMP Metadata (uses the values from hyperref)
\usepackage{hyperxmp}

% Make thumbnails (optional)
% \usepackage{thumbpdf}


%--------------------------------------------------
% Other settings
%--------------------------------------------------

% Define commands that appear not to eat spaces (optional)
\usepackage{xspace}
