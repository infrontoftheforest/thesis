% !TEX root = ../main.tex
\chapter{Introduction}
\label{chapter:Introduction}

% Motivation
In many fields, computer simulations can replace or supplement physical experiments. Through the mathematical modeling of physical problems from scientific and engineering disciplines, systems of partial differential equations are required to be solved. The discretization of these continuous systems by various means allows the solution to be approximated with a numerical method. In this thesis the focus is on solving the large sparse linear systems of equations that arise from discretization by methods such as the finite element method.

The development and implementation of efficient solution methods is essential to solving larger and more intricate problems. Direct methods can be appealing due to the black-box nature and ease of use, however, with larger and larger systems, the computational and memory resources required tend to be limiting factors for these methods. Although direct methods can arrive at a solution in a finite amount of steps, they do not scale optimally for such large systems. Certain iterative methods, while harder to tune and optimize, can achieve better scaling, and can make efficient use of increasingly-important parallel computing resources.

% Problem types to solve
This thesis will focus on the application of Krylov subspace methods, which in the optimal case are preconditioned to improve the rate of convergence, even for ill-conditioned linear systems. One of the most effective methods available for preconditioning Krylov methods are multigrid methods. Multigrid methods were developed as an effective method to deal with the shortcomings of more simple relaxation-based iterative methods, which is that they do not quickly and effectively reduce the low-oscillatory error components. The general idea behind multigrid methods is to represent the error which relaxation methods fail to eliminate on a coarser grid, such that it is more oscillatory and can be more effectively reduced. Different types of multigrid methods exist, and all require an understanding of the underlying principles to obtain the desired performance. Without this understanding, it is difficult to choose the correct parameters to obtain the performance these methods are known for, namely linear scaling with the number of unknowns and ability to solve ill-conditioned problems. The next section will focus on the development of geometric multigrid methods which require multiple discretizations to represent the problem on a coarser and coarser grid. This will then lead into a discussion of the aggregation algebraic multigrid methods, which can solve systems with unstructured meshes, or even without a geometric discretization. These multigrid methods will be used in the preconditioning of iterative methods known as Krylov subspace solvers in order to obtain robust and fast convergence for 3D elasticity problems in mechanics.
 